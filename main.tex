\documentclass[10pt, twoside]{article}
\usepackage{main}

% Aquí empieza el documento{{{
\begin{document}

%\maketitle
\thispagestyle{fancy}

\textbf{Alberto Oporto Ames: \#139}

\section{Preguntas}%
\label{sec:preguntas}

\begin{enumerate}
	\setcounter{enumi}{3}
	\item ¿Cómo funciona un transformador?
	\setcounter{enumi}{4}
	\item ¿Para que sirve el galvanómetro y cómo funciona?
\end{enumerate}

\section{Problemas}%
\label{sec:problemas}

\textbf{
	En el instante mostrado, se suelta el imán.
	Responde las siguientes preguntas respecto al observador:
}

\begin{figure}[H]
	\centering
	\begin{tikzpicture}[scale=1, transform shape]
		\foreach \y / \p in {0/$S$, -1/$N$}
		{
			\draw (0,\y) rectangle ++(1,1) node[pos=.5] {\p};
		}
		\draw[->] (2,0) -- ++(0,-2) node[midway, right] {$g$};

		\begin{scope}[canvas is xz plane at y=-3]
			\draw (0.5,0) circle (1);
		\end{scope}

		\draw (0.5, -5) -- +(45:1);
		\draw (0.5, -5) -- +(135:1);
	\end{tikzpicture}
\end{figure}
\begin{enumerate}[label=\alph*)]
	\item ¿Cuál es el sentido de la corriente inducida cuando el imán se
		aproxima a la espira?
		\subitem Horario.
	\item ¿Cuál es el sentido de la corriente inducida cuando el imán está al
		medio de la espira?
		\subitem No
	\item ¿Cuál es la dirección de la fuerza magnética inducida cuando el
		imán se aleja de la espira?
		\subitem Antihorario.
	\item Realiza una gráfica de $Fem$ inducida vs tiempo.
\end{enumerate}

\textbf{
	Se muestra el perfil de una espira conductora que gira uniformemente alrededor
	de un eje perpendicular al plano del papel.
}
\begin{figure}[H]
	\centering
	\begin{tikzpicture}[scale=1, transform shape]
		\coordinate (A) at (-1.75, 1.5);

		\draw (A) -- ++(-35:3);
		\draw[dashed] (A) -- ++(0,{sin(-35)*3});

		\draw[->] (-2,1) -- (2,1) node [pos=0.975, above=0.1cm] {$\vec{B}$};
		\draw[->] (-2,0) -- (2,0);
	\end{tikzpicture}
\end{figure}
\begin{enumerate}[label=\alph*)]
	\item Analiza el flujo magnético antes, después y en el instante que la
		espira se coloque en forma vertical.
	\item Analiza el sentido de la corriente inducida para el observador antes y
		después que la espira pase por el eje vertical.
\end{enumerate}

\textbf{
	\boldmath
	Se tiene una espira metálica dentro de un campo magnético uniforme de
	$0.5T$.
}
\tikzset
{
	equis/.pic =
	{
		%\node at (0,0) {\texttt{x}};
		\foreach \t in {45, 135, 225, 315}
		{
			\draw (0,0) -- (\t:0.125);
		}
	}
}
\begin{figure}[H]
	\centering
	\begin{tikzpicture}[scale=1, transform shape]
		\draw (0,0) circle (2);
		\foreach \x in {0, ..., 5}
		{
			\foreach \y in {0, ..., 5}
			{
				\pic at(\x-2.5,\y-2.5) {equis};
			}
		}
		\draw (-2.5, 2.5) circle (0.2) node [above=0.2cm] {$\vec{B}=cte$};
	\end{tikzpicture}
\end{figure}

\begin{enumerate}[label=\alph*)]
	\item Determina la fuerza electromotriz inducida y el sentido de la
		corriente inducida, si la espira empieza a dilatarse a razón de
		$8 \frac{cm^2}{s} $
		\begin{align*}
			|\mathcal{E}| &= \frac{d\phi}{dt}\\
			\\
			\frac{d\phi}{dt} &= 0.5T* 0.0008 \frac{m^2}{s} * \cancel{\cos{90°}}\\
			\\
			|\mathcal{E}| &= 4*10^{-4}
		\end{align*}
		Antihorario
	\item Determina la fuerza electromotriz inducida y el sentido de la
		corriente inducida, si la espira empieza a encogerse a razón de
		$10 \frac{cm^2}{s} $
		\begin{align*}
			|\mathcal{E}| &= \frac{d\phi}{dt}\\
			\\
			\frac{d\phi}{dt} &= 0.5T* 0.001 \frac{m^2}{s} * \cancel{\cos{90°}}\\
			\\
			|\mathcal{E}| &= 5*10^{-4}
		\end{align*}
		Horario
\end{enumerate}

\end{document}
%}}}
